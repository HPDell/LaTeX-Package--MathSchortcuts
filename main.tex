\documentclass{article}

\usepackage{mathshortcuts}
\usepackage[margin=0.5in]{geometry}
\usepackage{booktabs}
\usepackage{tcolorbox}
\tcbuselibrary{documentation,minted}
\tcbset{
    listing engine=minted,
    doc head command={
        interior style=fill,
        colback=yellow!20!white
    }
}

\title{Math Shortcuts}
\author{Yigong Hu}
\date{ }

\begin{document}

\maketitle

\section{Shortcut For Letters}

\begin{table}[h]
    \centering
    \begin{tabular}{ll|ll|ll|ll}\toprule
    \multicolumn{4}{c|}{\textbf{Alphabet}} & \multicolumn{4}{c}{\textbf{Greek}} \\\midrule
    \verb"\va" & $\va$ & \verb"\vA" & $\vA$ & \verb"\valpha" & $\valpha$ & \\
    \verb"\vb" & $\vb$ & \verb"\vB" & $\vB$ & \verb"\vbeta" & $\vbeta$ & \\
    \verb"\vc" & $\vc$ & \verb"\vC" & $\vC$ & \verb"\vchi" & $\vchi$ & \\
    \verb"\vd" & $\vd$ & \verb"\vD" & $\vD$ & \verb"\vdelta" & $\vdelta$ & \verb"\vDelta" & $\vDelta$ \\
    \verb"\ve" & $\ve$ & \verb"\vE" & $\vE$ & \verb"\vepsilon" & $\vepsilon$ & \\
    \verb"\vf" & $\vf$ & \verb"\vF" & $\vF$ & \verb"\veta" & $\veta$ & \\
    \verb"\vg" & $\vg$ & \verb"\vG" & $\vG$ & \verb"\vgamma" & $\vgamma$ & \verb"\vGamma" & $\vGamma$ \\
    \verb"\vh" & $\vh$ & \verb"\vH" & $\vH$ & \verb"\vtheta" & $\vtheta$ & \verb"\vTheta" & $\vTheta$ \\
    \verb"\vi" & $\vi$ & \verb"\vI" & $\vI$ & \verb"\viota" & $\viota$ & \\
    \verb"\vj" & $\vj$ & \verb"\vJ" & $\vJ$ & & \\
    \verb"\vk" & $\vk$ & \verb"\vK" & $\vK$ & \verb"\vkappa" & $\vkappa$ & \\
    \verb"\vl" & $\vl$ & \verb"\vL" & $\vL$ & \verb"\vlambda" & $\vlambda$ & \verb"\vLambda" & $\vLambda$ \\
    \verb"\vm" & $\vm$ & \verb"\vM" & $\vM$ & \verb"\vmu" & $\vmu$ & \\
    \verb"\vn" & $\vn$ & \verb"\vN" & $\vN$ & & \\
    \verb"\vo" & $\vo$ & \verb"\vO" & $\vO$ & \verb"\vomega" & $\vomega$ & \verb"\vOmega" & $\vOmega$ \\
    \verb"\vp" & $\vp$ & \verb"\vP" & $\vP$ & \verb"\vphi" & $\vphi$ & \verb"\vPhi" & $\vPhi$ \\
    \verb"\vq" & $\vq$ & \verb"\vQ" & $\vQ$ & \verb"\vpi" & $\vpi$ & \verb"\vPi" & $\vPi$ \\
    \verb"\vr" & $\vr$ & \verb"\vR" & $\vR$ & \verb"\vrho" & $\vrho$ & \\
    \verb"\vs" & $\vs$ & \verb"\vS" & $\vS$ & \verb"\vsigma" & $\vsigma$ & \verb"\vSigma" & $\vSigma$ \\
    \verb"\vt" & $\vt$ & \verb"\vT" & $\vT$ & \verb"\vtau" & $\vtau$ & \\
    \verb"\vu" & $\vu$ & \verb"\vU" & $\vU$ & \verb"\vupsilon" & $\vupsilon$ & \verb"\vUpsilon" & $\vUpsilon$ \\
    \verb"\bv" & $\bv$ & \verb"\vV" & $\vV$ & & \\
    \verb"\vw" & $\vw$ & \verb"\vW" & $\vW$ & \verb"\vpsi" & $\vpsi$ & \verb"\vPsi" & $\vPsi$ \\
    \verb"\vx" & $\vx$ & \verb"\vX" & $\vX$ & \verb"\vxi" & $\vxi$ & \verb"\vXi" & $\vXi$ \\
    \verb"\vy" & $\vy$ & \verb"\vY" & $\vY$ & & \\
    \verb"\vz" & $\vz$ & \verb"\vZ" & $\vZ$ & \verb"\vzeta" & $\vzeta$ & \\
    \bottomrule
    \end{tabular}
    \caption{Shortcut for letters}
    \label{tab:schortcut-letters}
\end{table}

\section{Others}

\begin{docCommands}{
    {doc name = T},
    {doc name = I},
    {doc name = IG}
}
These commands put some marks on a matrix to indicate transposed matrix, inverse matrix and generalised inverse matrix. They must be used in math mode and following something that can have superscripts.
\begin{dispExample}
For matrix $A$, its transposed matrix is $\vA\T$, its inverse matrix is $\vA\I$, 
and its generalised inverse matrix is $\vA\IG$.
\end{dispExample}
\end{docCommands}

\begin{docCommands}{
    {doc name = rvec, doc parameter = \marg{content}\oarg{suffix}},
    {doc name = cvec, doc parameter = \marg{content}\oarg{suffix}}
}
These commands quickly add a round bracket to row vector elements or column vector elements with a transpose mark.
\begin{dispExample}
Row vector $\vA = \rvec{\vA_{n_1}\T,\vA_{n_2}\T,\cdots,\vA_{n_m}\T}$;
column vector $\vA = \cvec{\vA_{n_1}\T,\vA_{n_2}\T,\cdots,\vA_{n_m}\T}$.
\end{dispExample}
The appearance of brackets will change in different math modes. 
In display mode, auto-increased brackets are used.
\begin{dispExample}
Row vector 
$$\vA = \rvec{\vA_{n_1}\T,\vA_{n_2}\T,\cdots,\vA_{n_m}\T}$$
column vector 
$$\vA = \cvec{\vA_{n_1}\T,\vA_{n_2}\T,\cdots,\vA_{n_m}\T}$$
\end{dispExample}
If something else like superscripts needs to be appended to a vector, use the optional argument.
\begin{dispExample}
Inverse of a matrix written in a vector: 
$$\vA = \rvec{\vA_{n_1}\T,\vA_{n_2}\T,\cdots,\vA_{n_m}\T}[\I]$$
\end{dispExample}
\end{docCommands}

\begin{docCommand}{rsquared}{\oarg{type}\marg{value}}
Show \rsquared{} with/without its value. The mandatory argument \meta{value} specify whether and what to show. The optional argument \meta{type} put subscript to \rsquared for use to show other related statistics.
\begin{dispExample}
Usually \rsquared{} is used to check fitness of regression models.
The diagnostic information shows that, \rsquared{0.88} which is high.
However, \rsquared{} usually increase with more variables.
So sometimes we use \rsquared[adj]{} which is usually less than \rsquared{}. 
In this model, \rsquared[adj]{0.85}.
\end{dispExample}
\end{docCommand}

\begin{docCommand}{AIC}{\oarg{type}\marg{value}}
    Show AIC with/without its value. The usage of this command is quite similar to that of \cs{rsquared}.
\end{docCommand}

\end{document}

